% Created 2025-11-14 Fri 09:45
% Intended LaTeX compiler: pdflatex
\documentclass[11pt]{article}
\usepackage[utf8]{inputenc}
\usepackage[T1]{fontenc}
\usepackage{graphicx}
\graphicspath{ {./img/} }
\usepackage{longtable}
\usepackage{wrapfig}
\usepackage{rotating}
\usepackage[normalem]{ulem}
\usepackage{amsmath}
\usepackage{amssymb}
\usepackage{capt-of}
\usepackage[hidelinks]{hyperref}
\usepackage[margin=2cm]{geometry}
\usepackage{csquotes}
\usepackage{subcaption}
\usepackage{enumitem}
\usepackage[sorting=none]{biblatex}
\usepackage{datetime}
\usepackage{xcolor}
\usepackage{tabularx}
\usepackage{booktabs}

% \date{\monthyeardate\today}
\title{AICE2004 Coursework Specification}

% \setlength{\parskip}{0em}
\makeatletter
\hypersetup{
 pdfauthor=\@author,
 pdftitle=\@title,
 pdfkeywords={},
 pdfsubject={},
 pdflang={English}}

\begin{document}

\maketitle

\tableofcontents

\section{Introduction}
\label{sec:intro}
\begin{itemize}
\item In this coursework, students are receive an unoptimised neural network written in pure C++, and improve it with optimisation, parallelisation and distribution using OpenMP, MPI and SLURM.
\item Students receive two code repositories
\begin{itemize}
\item \texttt{coursework}: Containing the neural network in C++.
\item \texttt{marking}: Containing the testing framework that guides the student and represents the quantitative part of the assessment.
\end{itemize}
\item This document outlines the procedures a student is expected to take to finish the coursework and the associated marking criteria alongside.
\end{itemize}



\newpage
\section{Problem Description}
\label{sec:problem}
The overall goal for this coursework is to take an unoptimised program that trains a 3-layer neural network, and improve its execution time via the material taught in the course.
This program is then tested using the "Testing" program, which builds, runs and evaluates the program according to the Quantitative section of the mark scheme.

\subsection{The Neural Network (NN) Program}
\label{sec:nn}
Repository: \url{https://git.soton.ac.uk/aice2004/coursework}

This repository contains a C++11 program that:
\begin{enumerate}
\item Trains a neural network using the Fashion-MNIST training dataset.
\item Evaluates the accuracy of trained network using the Fashion-MNIST testing dataset.
\item Saves the predictions of the testing dataset to the filesystem, which is then evaluated by the "Testing" program.
\end{enumerate}

Students should take this program and 1. \textbf{optimise} the \emph{code}, 2. \textbf{parallelise} it across multiple \emph{threads} and 3. \textbf{distribute} across multiple \emph{tasks}.
The neural network:
\begin{itemize}
\item Is unoptimised (simple code, single-threaded, all in one task).
\item Has 784 input neurons.
\item Parametrizable amounts of hidden neurons (with parameter \texttt{-z})
\item 10 output neurons representing the classes of the Fashion-MNIST dataset.
\end{itemize}

The program accepts a set of arguments that represent the neural network's hyperparameters.
These hyperparameters describe the network structure, how it is trained, and how it is executed on the hardware.
The available hyperparameters (arguments) are:
\begin{itemize}
\item Random Seed (e.g. \texttt{-s 42})
\item Epochs (\texttt{-e 10})
\item Learning Rate (\texttt{-l 0.001})
\item Batch Size (\texttt{-b 32})
\item Hidden Layer Size (\texttt{-z 100})
\item Threads per task (\texttt{-{}-{}threads 1})
\item Tasks (\texttt{-{}-{}tasks 4})
\end{itemize}

Instructions on getting started with this code are located in the \texttt{README.md} file at the root of this repository.

Students are provided the "Contract" (Section \ref{sec:contract}) of this document to ensure that the C++ program complies with the assumptions of the Testing one.
\subsection{The Testing Program}
\label{sec:testing}
Repository: \url{https://git.soton.ac.uk/aice2004/marking}

This repository contains the program that will determine the mark of each student according to the "Quantitative" section of the Marking scheme.
There are two versions of this program:
\begin{enumerate}
\item Internal marking: This is the version we keep to determine Quantitative mark on our end.
\item External testing: This is the version students get to iteratively solve this coursework.
\end{enumerate}

The program contains hyperparameter sets, which are combinations of arguments supplied to the program, such as number of epochs (\texttt{-e}), random seed (\texttt{-s}), or threads (\texttt{--threads}).
The only difference between the two versions described above are the provided hyperparameter sets, the code remains the same.

Students are expected to follow a \emph{Test-Driven Development} workflow, with the help of this repository.
This repository contains Python code that goes through these Steps:
\begin{enumerate}
\item Builds the program.
\item Runs the program and collects runtime statistics (execution time, threads and tasks allocated, etc.)
\item Across the entire provided hyperparameter set, evaluates the program according to the 'Quantitative' mark scheme (Section \ref{sec:quant}).
\end{enumerate}
For Steps 1 and 2, information is collected that is used in Step 3, therefore it is \textbf{crucial that you do not change the C++ program's reading/writing logic}.

Students are provided the "Contract" (Section \ref{sec:contract}) of this document to ensure that the C++ program complies with the assumptions of the Testing one.

\newpage
\section{Marking}
\label{sec:marking}
\subsection{Quantitative (60\%)}
\label{sec:quant}
The quantitative mark comes from the Testing program (Section \ref{sec:testing}), which follows the mark scheme described below.
For all internally tested hyper-parameters sets, the following should apply.
\begin{table}[h]
\centering
\begin{tabularx}{\textwidth}{lXr}
\toprule
\textbf{Mark \#} & \textbf{Description} & \textbf{Mark Awarded} \\
\midrule
BUILD1 & The program compiles with all provided hyperparameter sets. & 1 \\
BUILD2 & The run command (\texttt{scripts/run.sh}) uses SLURM to run it. & 2 \\
\midrule
RUN1 & The program executes successfully (exit code 0). & 2 \\
\midrule
RES1 & The accuracy is within ±5\% of the un-optimised network. & 10 \\
RES2 & The thread count at runtime is equal to the \texttt{{-}{-}threads} argument. & 10 \\
RES3 & The task count at runtime is equal to the \texttt{{-}{-}tasks} argument. & 10 \\
RES4 & (\texttt{{-}{-}threads=1, {-}{-}tasks=1}) execution time sped up by 0.5x\textsuperscript{\textdagger}. & 10 \\
RES5 & (\texttt{{-}{-}threads=N, {-}{-}tasks=1}) execution time scales linearly ($N \in [1, \infty]$). & 7.5 \\
RES6 & (\texttt{{-}{-}threads=X, {-}{-}tasks=N}) spawns N MPI processes. & 7.5 \\
\midrule
\multicolumn{2}{r}{\textbf{Total}} & \textbf{60} \\
\bottomrule
\end{tabularx}
\medskip
\small{\textsuperscript{\textdagger} Subject to change, will not become more difficult.}
\end{table}

\subsection{Qualitative (40\%)}
\label{sec:qual}
Students are to submit a maximum 2-page report detailing:
\begin{enumerate}
\item Performance issues with the original code.
\item How they addressed those performance issues with:
\begin{enumerate}
\item Optimising the code.
\item Parallelising with threads.
\item Distributing with tasks.
\end{enumerate}
\item Big O analysis of original \& improved code.
\end{enumerate}


\noindent Mark scheme:
\begin{table}[h]
\centering
\begin{tabularx}{\textwidth}{lXr}
\toprule
\textbf{Mark \#} & \textbf{Description} & \textbf{Total marks} \\
\midrule
REP1 & All major problems in unoptimised code have been identified. & 5 \\
REP2 & Justifications for problems in original program have been well explained. & 5 \\
REP3 & Big O analysis with 10,000 epochs and 1000 hidden neurons are reasonable and close to internal estimate (based on lectures). & 7.5 \\
REP4 & Sufficient C++ optimisations identified (according to lectures). & 5 \\
REP5 & Sufficient thread parallelisations identified (according to lectures). & 5 \\
REP6 & Sufficient task distribution strategies identified (according to lectures). & 5 \\
REP7 & Big O analysis of the improved program with 10,000 epochs and 1000 hidden neurons are reasonable (according to lectures). & 7.5 \\
\midrule
\multicolumn{2}{r}{\textbf{Total}} & \textbf{40} \\
\bottomrule
\end{tabularx}
\end{table}


\newpage
\section{Contract}
\label{sec:contract}
Students must oblige to the following constraints on the program to ensure that the internal testing utility is able to detect their changes.

\textbf{Note to students}: The default program fits all of these requirements.
The \texttt{README.md} file in the coursework repository's root contains advice on not breaking this contract.

\begin{enumerate}
\item script/build.sh
\begin{enumerate}
\item This \textbf{MUST}, without any arguments, build the \texttt{build/main} with the provided conda environment.
\end{enumerate}
\item build/main
\begin{enumerate}
\item This \textbf{MUST} be built by build.sh
\item \label{it:arg} This \textbf{MUST} take the following arguments:
\begin{verbatim}
-s, --random_seed <num>    Random seed (unsigned int).
-e, --epochs <n>           Number of epochs.
-l, --learning_rate <f>    Learning rate.
-b, --batch_size <n>       Batch size.
-z, --hidden_size <n>      Hidden layer size.
    --threads <n>          Number of threads per task (default 1).
    --tasks <n>            Number of tasks (default 1).
-h, --help                 Show this help message
\end{verbatim}
\end{enumerate}

\item script/run.sh
\begin{enumerate}
\item This \textbf{MUST} contain the required commands to run `build/main` on Iridis with the above arguments
\end{enumerate}

\item Outputs (\textbf{MUST})
\begin{enumerate}
\item \textbf{MUST} output into 'output' directory of under the project root.
\item \textbf{MUST} contain the Prediction:
\item \textbf{MUST} be saved to a directory with this format name:
\texttt{seed<n>-epochs<n>-lr<f>-batch<n>-hidden<n>-threads<n>-tasks<n>}
\begin{itemize}
\item Where \texttt{<n>} is an integer and \texttt{<f>} is a real number that corresponds to the arguments described above.
\end{itemize}
\end{enumerate}

\item All internal calculations \textbf{MUST} be at least 32bits (>=float)

\item "The network \textbf{MUST} stay within \textpm{}5\% of accuracy after optimisation"
\begin{itemize}
\item Subject to change, it will only be relaxed (not get harder).
\end{itemize}

\item "To improve the program, you \textbf{MUST} use OpenMP, MPI and SLURM."
\item "The program \textbf{MUST} compile on Iridis with the provided conda environment."
\begin{itemize}
\item With all warnings turned on in the compiler (\texttt{-Wall}), must produce no warnings.
\item We reserve the right to fix the GCC and C++ version once Iridis X migration has completed.
\item The CMakeLists.txt file will be changed accordingly.
\end{itemize}
\end{enumerate}

\section{Advice}
\label{sec:advice}
\begin{itemize}
\item Remember: this is NOT a neural networks course. You don't have to understand the maths, but only the computations so that you can optimise them.
\begin{itemize}
\item Read the neural network crash course in the \texttt{coursework} repository, it gives you a quick oversight on the computations.
\end{itemize}
\item To help you optimise your code, you can use profilers (search: linux profiler, also look at the end of Low Level Programming).
\item You will notice at first that Iridis runs the network slower than even your laptop. This changes quickly once you scale to many more threads and tasks.
\item If there are bugs or ambiguities, please report them via the issue submission system at \url{https://git.soton.ac.uk/aice2004/coursework/-/issues}.
\item If you're feeling overwhelmed:
\begin{enumerate}
\item You're not alone, and you can talk to other people (just don't share code).
\item It seems complicated to start with, but it's all just code, and you know how to code.
\end{enumerate}
\end{itemize}
\end{document}

%%% Local Variables:
%%% mode: LaTeX
%%% TeX-master: t
%%% End:
